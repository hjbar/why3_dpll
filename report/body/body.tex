% ------------------------------------------------------------------------------------------

\section{Introduction}

L'objectif de ce projet est d'implémenter un algorithme DPLL pour 3-SAT vérifié avec \whythree\ comme décrit dans le \sujet. L'archive contenant le code source est disponible sur \github.

Pour ce faire, nous avons dû compléter le squelette de code fourni, c'est-à-dire compléter la définition des fonctions \scan\ et \dpll\ du fichier \texttt{dpll.mlw}. Ainsi, dans les sections suivantes, on présentera l'implémentation et la vérification de ces fonctions.

% ------------------------------------------------------------------------------------------
% ------------------------------------------------------------------------------------------

\section{Fonction \scan}

\subsection{Implémentation}

En me basant sur les indications du \sujet\ pour la fonction \scan, je propose cette implémentation:
\begin{minted}{ocaml}
let scan (mm: assignment) (cl: array cls) (nv na nc: int) : (b: bool, mc: int)
=
  let ref i = 0 in
  let ref mc = nc in

  while i < mc do
    let (l1, l2, l3) = cl[i] in

    if (abs l1 < na && mm[abs l1] <> l1) ||
       (abs l2 < na && mm[abs l2] <> l2) ||
       (abs l3 < na && mm[abs l3] <> l3)
    then begin
      mc <- mc - 1;
      swap cl i mc
    end
    else if (abs l1 < na && mm[abs l1] = l1) &&
            (abs l2 < na && mm[abs l2] = l2) &&
            (abs l3 < na && mm[abs l3] = l3)
    then
      return (false, mc)
    else
      i <- i + 1
  done;

  (true, mc)
\end{minted}

\subsection{Pré-condition}

TODO

\subsection{Post-condition}

TODO

\subsection{Variant}

TODO

\subsection{Loop-invariant}

TODO

% ------------------------------------------------------------------------------------------
% ------------------------------------------------------------------------------------------

\section{Fonction \dpll}

\subsection{Implémentation}

En me basant sur les indications du \sujet\ pour la fonction \dpll, je propose cette implémentation:
\begin{minted}{ocaml}
let rec dpll (mm: assignment) (cl: array cls) (nv na nc: int) : (s: bool)
=
  let (b, mc) = scan mm cl nv na nc in

  if mc = 0 then true
  else if (not b) || (na = nv) then false
  else begin
    mm[na] <- na;

    if dpll mm cl nv (na + 1) mc then true
    else begin
      mm[na] <- -na;
      dpll mm cl nv (na + 1) mc
    end
  end
\end{minted}

\subsection{Pré-condition}

TODO

\subsection{Post-condition}

TODO

\subsection{Variant}

TODO

\subsection{Loop-invariant}

TODO

% ------------------------------------------------------------------------------------------
% ------------------------------------------------------------------------------------------

\section{Conclusion}

\subsection{Synthèse}

TODO

\subsection{Problèmes rencontrés}

TODO

\subsection{Remarques}

TODO

% ------------------------------------------------------------------------------------------
